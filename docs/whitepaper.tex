\documentclass{article}
\usepackage{graphicx}
\usepackage{hyperref}
\hypersetup{
    colorlinks=true,
    linkcolor=blue,
    filecolor=magenta,      
    urlcolor=cyan,
}
\begin{document}

\title{Cookie Injection but with Real Cookies}
\author{Jake Vossen, Fisher Darling}

\maketitle

\begin{abstract}
  
\textit{Cookie Injection but with Real Cookies} is a puzzle game
designed to teach the basics of web exploitation to $3^{rd}$ through
$6^{th}$ grades through the analogy of actual cookies. The game is
broken down into several \textbf{stages}, each teaching a important
lesson in cybersecurity. This allows us to explore cookie injection,
MITM attacks, encryption, SQL/template injection and other security
exploitations. At each stage, the player must bake a cookie that is
pleasing to the \textbf{cookie monster}. In this scenario, the player
represents a computer client, while the cookie monster plays the role
an authentication server. Upon proper presentation of a cookie, the
player can move on to the next stage. All external assets and code is
either public domain or licensed under the open source MIT or GPL
licenses, including the \href{https://godotengine.org}{Godot} game
engine.



\end{abstract}

\section{Inspiration}

Our goal is to teach children from about the $3^{rd}$ grade to
$7^{th}$ grade about critical thinking skills involved in the cybersecurity
field. While some of the concepts are (hopefully) not prevalent in
production today, teach the same type of problem solving mindset that
needs to be taught. By using cookies analogously with HTTP requests,
we can learn about many concepts that will be handy as the player
progresses to CTFs and hopefully to a larger carrier in the cyber
industry. 

\section{Example Stages}

Listed below are a few example stages. These are not \textit{all} of
the stages, merly a subset that we find notable for reaching a
landmark or obtaining a learning objective.

The \textbf{difficulty} defines at what approximate age will the stage
be trivial. This is not at all the minimum age to be able to accomplish
the level with understanding, a much younger child will be able to do
this, just less trivially. These are also our best estimates, not a
scientific analysis of difficulty.

Each stage has an \textbf{learning objective}, which is the goal that
the level is trying to teach. The objective is cybersecurity concept /
idea that the player should obtain after completing the level.

Additionally, each stage has a \textbf{learning significance},
explaining why the concept taught is important in the world of
security. While sometimes these will feel far fetched from the basic
actions preformed by the user, the important part is giving the player
some concept (even if it is subconscious) about the important security
issues facing the web today.

\subsection{Introduction | Difficulty: Third Grade}

Stage one is the most basic of stages. The player simply goes up to
the cookie monster, who then displays a message requesting a chocolate
chip cookie. The player has to go to the oven and make a simple
chocolate chip cookie, which he can give to the cookie monster to pass
through the stage.

\subsubsection{Learning Objective}

\begin{itemize}

  \item Introduce the player to the dialog, controls, and play style of
the game
  \item Provide an example of a successful stage
  \item Provide an example of a situation where you are successfully
logging into a website with no web exploitation used.
\end{itemize}

\subsubsection{Learning Significance}

\begin{itemize}
  \item Introduce the idea of the client server model
  \item Introduce the idea of payloads from the client to the server,
    and the server preforming actions based of that request
  \item Introduce the idea of what an API call is
\end{itemize}

\subsection{Fakeout | Difficulty: Third Grade}

This stage introduces the idea of not using cookies as instructed. When the player
walks up to the cookie monster, he claims that he is the biggest
peanut butter cookie fan in the world, and he wants you to give him a
peanut butter cookie. He gives you instructions on how to make a
peanut butter cookie. However, behind and around the cookie monster there are
signs proclaiming his love for oatmeal raisin cookies. The cookie
monster only lets you pass if you give him a oatmeal raisin cookie,
and will reject your cookie if you give him a peanut better cookie. 

\subsubsection{Learning Objectives}

\begin{itemize}
  \item Web exploitations often include disobeying instructions given
    to the website
  \item Knowing background information about the website (such as
    programming language, database format, etc) can provide powerful
    insight into it's vulnerabilities
  \item The client cannot be trusted to provide 'safe' input
  \item Basic \textit{admin = true} type exploitation on simple sites
\end{itemize}

\subsubsection{Learning Significance}

\begin{itemize}
  \item Almost every exploit comes from a user not obeying the
    instructions of the server
  \item Websites that use cookies have to be incredible careful with
    how they encode / encrypt the data
  \item This is a similar setup to SQL injection, a classic example of
    web exploitation
\end{itemize}

\subsection{MITM (Monster in the Middle) | Difficulty: 5th Grade}

When the player approaches the cookie monster, the cookie monster wont
give him any information about how to pass. However, he can watch
other people go through with the correct cookies. Luckily, there is a
empty shell of a cookie monster, and the player can setup another
stopping point for the automated visitors to go through. The visitors
think that he is the monster, and gives the player the cookie when
they ask. Using the cookie given to him, he can then go to the
\textit{real} cookie monster with the correct recipe and pass the
stage.

\subsubsection{Learning Objectives}

\begin{itemize}
  \item Concepts behind man in the middle attacks
  \item Critical thinking regarding obtaining authentication tokens
  \item Realize the importance of cryptography as those keys are
    introduced.
\end{itemize}

\subsubsection{Learning Significance}

\begin{itemize}
  \item Most people understand that HTTPS is important, but this
    starts a great seed of \textit{why} it is important
  \item It helps students realize the importance of trust in a
    security environment
\end{itemize}

\pagebreak

\subsection{Advanced MITM | Difficulty: 5th Grade}

After your shenanigans with setting up your fake outpost, the monster
and the visitors decided to get tricky on you. The monster and the
visitors set up a strategy of shifting around the ingredients in the
recipe so you don't know what cookie to make. The shift looks like
this: %penut -> raisens, raisens -> chocolate chip -> oatmeal ->
% raisens
\begin{table}[h]
\centering
\begin{tabular}{l|l}
\textbf{Original}       & \textbf{Encrypted}      \\ \hline
Oatmeal        & Raisins        \\ \hline
Raisins        & Chocolate Chip \\ \hline
Chocolate Chip & Oatmeal        \\
\end{tabular}
\end{table}

\subsubsection{Learning Objectives}

\begin{itemize}
  \item The concepts and ideas of encryption
  \item Re-inforcing MITM attacks
  \item Introduce the idea of sources of truth (Certificate
    Authorities)
\end{itemize}

\subsubsection{Learning Significance}

\begin{itemize}
  \item This semi-advanced problem will help bridge the gap between
    theoretical games and actual security best practices
  \item Prepare them for problems they will see in middle/high school CTFs
\end{itemize}

\section{From MVP to Full Deployment}

Here is a layout/cost estimate of bringing this game to a MVP that
could be used by students to improve their cybereducation.

\subsection{Overview}

\begin{table}[h]
\begin{tabular}{l|l}
\textbf{Task}                       & \textbf{Man-Hours (assuming experince)} \\ \hline
Full implementation of necessary UI & 35                                                 \\ \hline
Sound Design                        & 25                                                 \\ \hline
Asset Redesign                      & 30                                              \\ \hline
Level Generation                    & 40                                                 \\ \hline
Tie-up                              & 30                                               \\ \hline
\textbf{Total}                      & \textbf{160}                                              \\ 
\end{tabular}
\end{table}

I will now go into further detail of explaining the estimates and the
sub-tasks that add up to that amount

\subsection{Full implementation of necessary UI}

While the game has basic features such as a next level button and the
ability to increment and decrement the ingredents, the vast majority
of actual UI design is yet to be done. There still remains:

\begin{itemize}
  \item Main menu
  \item Pause menu
  \item Clean transition level buttons
  \item Selection of + and - can be improved
\end{itemize}

I figured that most of those things would take a handful of hours for
someone to create and implement fully, so 35 hours is a reasonable
estimate.

\subsection{Sound Desing}

The game currently has no sound, which means that every little sound
will have to be created / bought / found.

These include, but are not limited to

\begin{itemize}
  \item Jumping
  \item Landing
  \item Selection on oven
  \item Baking
  \item Text scrolling (can probably find a sound for this)
  \item Walking
\end{itemize}

\subsection{Asset Redesign}

While we where creating the game, we made use of (MIT Licensed) assets
that came with the Godot engine. The ``cookie monster'' is actually
just a large enemy from the platfromer demo, for example. A team would
have to create assets for at least the player and the cookie
monster. It would be ideal to get the license for the sesame street
cookie monster to use in our game, although I am not sure what the
details of that would be.

\subsection{Level Generation}

We did not have time to make nearly the number of levels that would be
ideal for a published game. Although, we do have a template and pretty
good scenes that allow for quick duplication and modification to
create new levels. The length of time spent on this will be entierly
up to the development team, depending on the number of lessions that
they could fit in.

\subsection{Tie-up}

Additionally to the tasks above, there would be a non-trivial amount
of time to tie everything up after the individual peices where
created. This could be a very quick process if someone is familiar
with the Godot engine, or other similar game engines. You can also
think of this as time to market/publish and work with partners and deployment.

\end{document}