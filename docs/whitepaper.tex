\documentclass{article}
\usepackage{graphicx}
\usepackage{hyperref}
\hypersetup{
    colorlinks=true,
    linkcolor=blue,
    filecolor=magenta,      
    urlcolor=cyan,
}
\begin{document}

\title{Cookie Injection but with Real Cookies}
\author{Jake Vossen, Fisher Darling}

\maketitle

\begin{abstract}
  
\textit{Cookie Injection but with Real Cookies} is a puzzle game
designed to teach the basics of web exploitation to $3^{rd}$ through
$6^{th}$ grades through the analogy of actual cookies. The game is
broken down into several \textbf{stages}, each teaching a important
lesson in cybersecurity. This allows us to explore cookie injection,
MITM attacks, encryption, SQL/template injection and other security
exploitations in a fun and engaging manner. At each stage, the player must bake a cookie that pleases the \textbf{cookie monster}. In this scenario, the player
represents a computer client, while the cookie monster plays the role
an authentication server. Upon proper presentation of a cookie, the
player can move on to the next stage. All external assets and code is
either public domain or licensed under the open source MIT
license, including the \href{https://godotengine.org}{Godot} game
engine.



\end{abstract}

\section{Inspiration and Scope}

Our goal is to teach children about $3^{rd}$ grade to $7^{th}$ grade
about the critical thinking skills involved in the cybersecurity
field. While some of the concepts are (hopefully) not prevalent in
production today, they represent the same type of problem solving
mindset that needs to be taught. By using cookies analogously with
HTTP requests, we can learn about many concepts that will be handy as
the player progresses to CTFs and hopefully to a larger carrier in the
cyber industry.

Another part of the scope is that this game can be as open as desired
by the publishers. We used no code or assets that put limitation or
restriction on the game, which means that students and developers can
fork, update, and improve upon \textit{Cookie Injection but with Real
Cookies}. This will allow for additional educational purpose, as the players
and educators can explore the source code, modifying as they wish.

\section{Example Stages}

Listed below are a few example stages. These are not \textit{all} of
the stages, merely a subset that we find notable for reaching a
landmark or obtaining a learning objective. Further development work
should be done on developing more stages.

The \textbf{difficulty} defines the approximate age that the stage
will be easily solvable. This is not at all the minimum age to be able to accomplish
the level with understanding, a much younger child will be able to do
this, just less trivially. These are also our best estimates, not a
scientific analysis of difficulty.

Each stage has a \textbf{learning objective}, which is the goal that
the level is trying to teach. The objective is a cybersecurity concept or
idea that the player should have a basic understanding for after completing the level.

Additionally, each stage has a \textbf{learning significance},
explaining why the concept taught is important in the world of
security. While sometimes these will feel far fetched from the basic
actions preformed by the user, the important part is giving the player
some \textbf{concept} (even if it is subconscious) about the important security
issues facing the web today.

\subsection{Introduction | Difficulty: Third Grade}

Stage one is the most basic of stages. The player simply goes up to
the cookie monster, who then displays a message requesting a chocolate
chip cookie. The player has to go to the oven and make a simple
chocolate chip cookie, which he can then give to the cookie monster to pass the stage.

\subsubsection{Learning Objective}

\begin{itemize}

  \item Introduce the player to the dialog, controls, and play style of
the game
  \item Provide an example of a successful stage
  \item Provide an example of a situation where you are successfully
logging into a website with no web exploitation used.
\end{itemize}

\subsubsection{Learning Significance}

\begin{itemize}
  \item Introduce the idea of the client server model. The player is the client
  providing the information, and the monster (server) is authenticating on that information.
  \item Introduce the idea of payloads from the client to the server. A payload
  is a cookie that can have a different number of ingredients.
    and the server preforming actions based of that request
  \item Introduce the idea of what an API call is.
\end{itemize}

\subsection{Fakeout | Difficulty: Third Grade}
% successful cookie is only oatmeal, he asks for raisens and chocolate chips
This stage introduces the idea of not using cookies as
instructed. When the player walks up to the cookie monster, the cookie monster claims
that he is the biggest chocolate chip and raisin fan in the world, and
he wants you to give him a chocolate chip and raisin cookie. However, behind
and around the cookie monster there are signs proclaiming his love for
oatmeal cookies. The cookie monster only lets you pass if you give him
a oatmeal cookie, and will appreciate your cookie if you give him a
chocolate chip rasin cookie, but not let you pass.

\subsubsection{Learning Objectives}

\begin{itemize}
  \item Web exploitations often include disobeying instructions given by the website.
  For this stage, the player (client) needs to disobey the monster's instructions in order
  to pass the stage.
  \item Knowing background information about the website (such as
programming language, database format, framework, etc) can provide powerful
insight into it's vulnerabilities. This stage requires the player to think critically about
the monster's environment to determine what the monster really wants.
  \item The client cannot be trusted to provide `safe' input. This is the first example
  of actual exploitation. The monster should have had stricter authentication requirements
  so that this exploitation could not have so easily occurred.
  \item Basic \textit{admin = true} type exploitation on simple sites (Cookie Injection).
\end{itemize}

\subsubsection{Learning Significance}

\begin{itemize}
  \item Almost every exploit comes from a user not obeying the
    instructions of the server.
  \item Websites that use cookies have to be incredibly careful with
    how they encode / encrypt the data.
  \item This is a similar setup to SQL injection, a classic example of
    providing input that is malicious.
\end{itemize}

\subsection{MITM (Monster in the Middle) | Difficulty: 5th Grade}

When the player approaches the cookie monster, the cookie monster will
give him any useful information about how to pass. However, he can watch
other people go through with the correct cookies. Luckily, there is a
empty shell of a cookie monster, and the player can setup another
stopping point for the automated visitors to go through. The visitors
think that he is the monster, and they give the player their cookie when
when asked. Using the cookie given to the player, the player can then go to the
\textit{real} cookie monster with the correct recipe and pass the
stage.

\subsubsection{Learning Objectives}

\begin{itemize}
  \item Concepts behind man in the middle attacks. The player acts as a
  \textit{Monster in the Middle} in order to view the recipe of the privileged cookies.
  \item Critical thinking regarding obtaining authentication tokens.
  \item Prompt the question: How can a client's information be secured in transit to server?
\end{itemize}

\subsubsection{Learning Significance}

\begin{itemize}
  \item Most people understand that HTTPS is important, but this
    plants a seed of \textit{why} it is important. Why is it so easy to
    masquerade as the monster? 
  \item It helps students realize the concepts and importance of trust in a
    security environment. 
\end{itemize}

\subsection{Advanced MITM | Difficulty: 5th Grade}

After your shenanigans with setting up your fake outpost, the monster
and the visitors decided to get tricky on you. The monster and the
visitors set up a strategy of shifting around the ingredients in the
recipe so you don't know what cookie to make. Your job is to
intercepts the messages that determine how they are going to shuffle
the ingredients, and re-create the cookies based off of the mapping and
the encrypted cookie recipes that travel with other visitors.

A shift could look like
this: %peanut -> raisins, raisins -> chocolate chip -> oatmeal ->
% raisins
\begin{table}[h]
\centering
\begin{tabular}{l|l}
\textbf{Original}       & \textbf{Encrypted}      \\ \hline
Oatmeal        & Raisins        \\ \hline
Raisins        & Chocolate Chip \\ \hline
Chocolate Chip & Oatmeal        \\
\end{tabular}
\end{table}

\subsubsection{Learning Objectives}

\begin{itemize}
  \item The concepts and ideas of encryption through the introduction of a simple
  substitution cipher.
  \item Re-enforcing MITM attacks and introduce  methods used to prevent them.
  \item Re-enforce the idea of sources of truth (Certificate
    Authorities, Signing Chains)
\end{itemize}

\subsubsection{Learning Significance}

\begin{itemize}
  \item This semi-advanced problem will help bridge the gap between
    theoretical games and actual security best practices and exploitations.
  \item Prepare students for problems they will see in middle/high school CTFs.
  \item The sharing of a substitution cipher among parties introduces the concept of
  a source of true and a shared symmetric key.
\end{itemize}

\section{Plan to MVP}

This section is dedicated to the process the game would go
though to get from where it is now to release with full
potential. Granted, the time estimates may be off since
we are inexperienced with planning long-term projects, however these estimates
are taken from a conservative standpoint.

\subsection{Overview}

\begin{table}[h]
\begin{tabular}{l|l}
\textbf{Task}                       & \textbf{Man-Hours (assuming experience)} \\ \hline
Full implementation of necessary UI & 35                                                 \\ \hline
Sound Design                        & 25                                                 \\ \hline
Asset Redesign                      & 30                                              \\ \hline
Level Generation and Design         & 70                                                 \\ \hline
Tie-up                              & 30                                               \\ \hline
\textbf{Total}                      & \textbf{190}                                              \\ 
\end{tabular}
\end{table}

We will now go into further detail of explaining the estimates and the
sub-tasks that add up to that amount

\subsection{Full implementation of necessary UI}

While the game has basic features such as a next level button and the
ability to increment and decrement the ingredients, the vast majority
of actual UI design is yet to be done. There still remains:

\begin{itemize}
  \item Main menu
  \item Pause menu
  \item Clean transition level buttons
  \item Selection of + and - can be improved
  \item Visualization of a Cookie
\end{itemize}

We figure that most of those things would take a handful of hours for
someone knowledgeable in the Godot engine to create and implement fully, 
so 35 hours is a reasonable estimate.

\subsection{Sound Design}

The game currently has no sound, which means that every little sound
will have to be created / bought / found.

These include, but are not limited to

\begin{itemize}
  \item Jumping
  \item Landing
  \item Selection on oven
  \item Baking
  \item Text scrolling (can probably find a sound for this)
  \item Walking
\end{itemize}

There are a lot of opensource sound assets that we discovered. We hope
that these assets could be used for the final development of the game.

\subsection{Asset Redesign}

While we were creating the game, we made use of MIT Licensed assets
that came with the Godot engine. The ``cookie monster'' is actually
just a large enemy from the platformer demo. The player and its animations 
is a character asset found on the opensource Godot Asset Library. A team would have to create assets for at least the 
player and the cookie monster. It would be ideal to get the license 
for the sesame street cookie monster to use in our game, although we am not sure how that
would work out.

\subsection{Level Generation}

We did not have time to make nearly the number of levels that would be
ideal for a published game. Although, we do have a template and scenes that allow for quick duplication and modification to
create new levels. The length of time spent on this will be entirely
up to the development team, and the educators on the project. 
The learning objectives for each level will need to be tested and discussed,
and a overall educational flow / objective would need to be created.
\subsection{Tie-up}

Additionally to the tasks above, there would be a non-trivial amount
of time to tie everything up after the individual pieces where
created. This could be a very quick process if someone is familiar
with the Godot engine, or other similar game engines. You can also
think of this as time to market/publish and work with partners / educators and
deployment.

\end{document}